% GENERAL DOCUMENT SETUP
\usepackage[top = 1.5cm,
            bottom = 1.5cm,
            left = 2cm,
            right = 2cm,
            headsep = 10pt,
            a4paper]{geometry}
\usepackage[english]{babel}
\usepackage[none]{hyphenat}
\usepackage[utf8]{inputenc}
\usepackage[T1]{fontenc}
\usepackage[parfill]{parskip}

% MISC TOOLS
\usepackage{xparse}
\usepackage{xspace}
\usepackage{calc}
\usepackage{listings}

\makeatletter
\@addtoreset{chapter}{part}
\makeatother
% Forcibly redefine a command, whether or not it exists
\newcommand{\forcecommand}[1]{\providecommand{#1}{}\renewcommand{#1}}

% LISTS, TABLES, FIGURES
\usepackage{enumitem}
    \setlist{nolistsep}
    \setlist[enumerate, 1]{label={\arabic*.}} % Default enumerate list is Arabic numbering
    \setlist[enumerate]{topsep=0pt}
\usepackage{tabularx}
\usepackage{booktabs}

% GENERAL GRAPHICS AND FIGURES
\usepackage{graphicx}
\usepackage{tikz}
    \usepackage{tikz-3dplot, tkz-graph, tkz-berge}
    \usepackage{tikz-cd}
\usepackage{pstricks, pst-node, pst-plot}
\usepackage{float}
\usepackage{subcaption}

% CUSTOM COLORS
\newlength{\tcbBorderWidth}\setlength{\tcbBorderWidth}{3pt}
\definecolor{definitionColor}{RGB}{31,117,254} % xcolor already loaded
\definecolor{resultColor}{RGB}{209,17,65}
\colorlet{resultBackColor}{black!5}
\definecolor{exampleColor}{RGB}{0,177,89}
\colorlet{exampleBackColor}{black!0}
\definecolor{exerciseColor}{RGB}{0,177,89}
\colorlet{exerciseBackColor}{black!0}
\colorlet{problemColor}{black!50}
\colorlet{problemBackColor}{black!10}
\definecolor{theme}{RGB}{31,117,254}


% MATH PACKAGES
\usepackage{amssymb}
\usepackage{amsmath}
\usepackage{amsthm}
\usepackage{amsfonts}
\usepackage{stmaryrd}
\usepackage{centernot}
\usepackage{bbm}
\usepackage{cancel}
\usepackage{mathtools}

% APPEARANCE
% Title formats
\usepackage[compact]{titlesec}
	\titleformat{\chapter}[hang]{\Huge\sffamily\bfseries}{\hspace*{-30pt}\thechapter\hspace{10pt}\textcolor{theme!50}{\raisebox{-2.5pt}{\HUGE|}}\hspace{10pt}}{0pt}{\Huge\sffamily\bfseries}
	\titlespacing*{\chapter}{0pt}{-30pt}{20pt}
	\titlespacing*{\section}{}{0pt}{20pt}
% Headers and footers
\usepackage{fancyhdr}
\pagestyle{fancy}
\renewcommand{\chaptermark}[1]{\markboth{\sffamily\normalsize\thechapter.\ #1}{}} % Chapter text font settings
\renewcommand{\sectionmark}[1]{\markright{\sffamily\normalsize\thesection\hspace{5pt}#1}{}} % Section text font settings
\fancyhf{}
\fancyhead[LE]{\leftmark} % Nearest section name on left side of even pages
\fancyhead[RO]{\rightmark} % Current chapter name on right side of odd pages
\renewcommand{\headrulewidth}{0.5pt} % Width of rule under header
\addtolength{\headheight}{2.5pt} % Increase spacing around header
\renewcommand{\footrulewidth}{0pt} % No footer rule
\fancypagestyle{plain}{\fancyhead{}\renewcommand{\headrulewidth}{0pt}} % Style for plain pages
\fancyfoot[LE,RO]{\sffamily\normalsize\thepage}
% Remove header from odd empty pages at end of chapters
\makeatletter
\renewcommand{\cleardoublepage}{
    \clearpage\ifodd\c@page\else
    \hbox{}
    \vspace*{\fill}
    \thispagestyle{empty}
    \newpage
    \fi}

% MARGIN NOTES
\usepackage{marginnote}
    \setlength\marginparwidth{2cm}
\usepackage{changepage}
    \strictpagecheck

\makeatletter
\newcommand{\marparleft}[1]{
\checkoddpage
\ifoddpage{\reversemarginpar\marginnote[#1]{}}
\else{\normalmarginpar\marginnote[#1]{}}
\fi
}

\newcommand{\marparright}[1]{
\checkoddpage
\ifoddpage{\normalmarginpar\marginnote{#1}}
\else{\reversemarginpar\marginnote{#1}}
\fi
}

\newcounter{LectureNumber}
\newcommand{\lecnumstyle}[1]{\marparleft{L{#1}}}
\newcommand{\lecturenumber}{\stepcounter{LectureNumber}\lecnumstyle{\theLectureNumber}}

% FONTS
\usepackage{avant} % Avant for headings
%\usepackage{sansmathsfonts}
%\usepackage{newpxtext,newpxmath}
%\renewcommand*\familydefault{\sfdefault}
\usepackage{moresize}
%\renewcommand{\sffamily}{\rmmfamily}

% BIBLIOGRAPHY AND INDEX
\usepackage[style = numeric,
            citestyle = numeric,
            sorting = nyt,
            sortcites = true,
            autopunct = true,
            autolang = hyphen,
            hyperref = true,
            abbreviate = false,
            backref = true,
            backend = biber]{biblatex}
%\defbibheading{bibempty}{}
\usepackage{csquotes}
\usepackage{makeidx} % To make index
    %\makeindex % Tells LaTeX to create index files

% SECTION NUMBERING IN MARGIN
\makeatletter
\renewcommand{\@seccntformat}[1]{\llap{\textcolor{theme}{\csname the#1\endcsname}\hspace{1em}}}                    
\renewcommand{\section}{\@startsection{section}{1}{\z@}
	{-4ex \@plus -1ex \@minus -.4ex}
	{1ex \@plus.2ex }
	{\normalfont\large\sffamily\bfseries\boldmath}}
\renewcommand{\subsection}{\@startsection {subsection}{2}{\z@}
	{-3ex \@plus -0.1ex \@minus -.4ex}
	{0.5ex \@plus.2ex }
	{\normalfont\sffamily\bfseries\boldmath}}
\renewcommand{\subsubsection}{\@startsection {subsubsection}{3}{\z@}
	{-2ex \@plus -0.1ex \@minus -.2ex}
	{.2ex \@plus.2ex }
	{\normalfont\small\sffamily\bfseries\boldmath}}                        
\renewcommand\paragraph{\@startsection{paragraph}{4}{\z@}
	{-2ex \@plus-.2ex \@minus .2ex}
	{.1ex}
	{\normalfont\small\sffamily\bfseries}}
%%% TABLE OF CONTENTS STUFF, DON'T TOUCH
\usepackage{titletoc}
\setcounter{tocdepth}{1}
\contentsmargin{0cm} % Removes the default margin
% Part text styling
\titlecontents{part}[0cm]
{\addvspace{20pt}\centering\large\bfseries}
{}
{}
{}
% Chapter text styling
\titlecontents{chapter}[1.25cm] % Indentation
{\addvspace{12pt}\large\sffamily\bfseries} % Spacing and font options for chapters
{\color{theme!60}\contentslabel[\Large\thecontentslabel]{1.25cm}\color{theme}} % Chapter number
{\color{theme}}  
{\color{theme!60}\normalsize\;\titlerule*[.5pc]{.}\;\thecontentspage} % Page number
% Section text styling
\titlecontents{section}[1.25cm] % Indentation
{\addvspace{3pt}\sffamily\bfseries} % Spacing and font options for sections
{\contentslabel[\thecontentslabel]{1.25cm}} % Section number
{}
{\hfill\color{black}\thecontentspage} % Page number
[]
% Subsection text styling
\titlecontents{subsection}[1.25cm] % Indentation
{\addvspace{1pt}\sffamily\small} % Spacing and font options for subsections
{\contentslabel[\thecontentslabel]{1.25cm}} % Subsection number
{}
{\ \titlerule*[.5pc]{.}\;\thecontentspage} % Page number
[]
% List of figures
\titlecontents{figure}[0em]
{\addvspace{-5pt}\sffamily}
{\thecontentslabel\hspace*{1em}}
{}
{\ \titlerule*[.5pc]{.}\;\thecontentspage}
[]
% List of tables
\titlecontents{table}[0em]
{\addvspace{-5pt}\sffamily}
{\thecontentslabel\hspace*{1em}}
{}
{\ \titlerule*[.5pc]{.}\;\thecontentspage}
[]


%%% MINI TABLE OF CONTENTS IN PART TITLE PAGE
% Chapter text styling
\titlecontents{lchapter}[0em] % Indenting
{\addvspace{5pt}\normalsize\sffamily\bfseries} % Spacing and font options for chapters
{\color{theme}\contentslabel[\large\thecontentslabel]{1.25cm}\color{theme}} % Chapter number
{}  
{\color{theme}\small\sffamily\bfseries\;\titlerule*[.5pc]{.}\;\thecontentspage} % Page number
% Section text styling
\titlecontents{lsection}[0em] % Indenting
{\sffamily\small} % Spacing and font options for sections
{\contentslabel[\thecontentslabel]{1.25cm}} % Section number
{}
{\color{black}\footnotesize\sffamily\;\titlerule*[.5pc]{.}\;\thecontentspage}
% Subsection text styling
\titlecontents{lsubsection}[.5em] % Indentation
{\normalfont\footnotesize\sffamily} % Font settings
{}
{}
{}

%TODO CHECK WHAT THIS DOES
% numbered part in the table of contents
\newcommand{\hsp}{\hspace{20pt}}
\newcommand{\@mypartnumtocformat}[2]{%
	\setlength\fboxsep{0pt}%
	\noindent\colorbox{theme!20}{\strut\parbox[c][.7cm]{\ecart}{\color{theme!70}\Large\sffamily\bfseries\centering#1}}\hskip\esp\colorbox{theme!40}{\strut\parbox[c][.7cm]{\linewidth-\ecart-\esp}{\Large\sffamily\centering#2}}}%
% unnumbered part in the table of contents
\newcommand{\@myparttocformat}[1]{%
	\setlength\fboxsep{0pt}%
	\noindent\colorbox{theme!40}{\strut\parbox[c][.7cm]{\linewidth}{\Large\sffamily\centering#1}}}%
\newlength\esp
\setlength\esp{4pt}
\newlength\ecart
\setlength\ecart{1.2cm-\esp}
\newcommand{\thepartimage}{}%
\newcommand{\partimage}[1]{\renewcommand{\thepartimage}{#1}}%
\def\@part[#1]#2{%
	\ifnum \c@secnumdepth >-2\relax%
	\refstepcounter{part}%
	\addcontentsline{toc}{part}{\texorpdfstring{\protect\@mypartnumtocformat{\thepart}{#1}}{\partname~\thepart\ ---\ #1}}
	\else%
	\addcontentsline{toc}{part}{\texorpdfstring{\protect\@myparttocformat{#1}}{#1}}%
	\fi%
	\startcontents%
	\markboth{}{}%
	{\thispagestyle{empty}%
		\begin{tikzpicture}[remember picture,overlay]%
		\node at (current page.north west){\begin{tikzpicture}[remember picture,overlay]%	
			\fill[theme!20](0cm,0cm) rectangle (\paperwidth,-\paperheight);
			\node[anchor=south east] at (\paperwidth-1cm,-\paperheight+1cm){\parbox[t][][t]{12cm}{
					\printcontents{l}{0}{\setcounter{tocdepth}{1}}%
			}};
			\node[anchor=north east] at (\paperwidth-1.5cm,-3.25cm){\parbox[t][][t]{15cm}{\strut\raggedleft\color{black}\fontsize{20}{20}\sffamily\bfseries#2}};
			\end{tikzpicture}};
\end{tikzpicture}}%
\@endpart}
\def\@spart#1{%
\startcontents%
\phantomsection
{\thispagestyle{empty}%
	\begin{tikzpicture}[remember picture,overlay]%
	\node at (current page.north west){\begin{tikzpicture}[remember picture,overlay]%	
		\fill[theme!20](0cm,0cm) rectangle (\paperwidth,-\paperheight);
		\node[anchor=north east] at (\paperwidth-1.5cm,-3.25cm){\parbox[t][][t]{15cm}{\strut\raggedleft\color{black}\fontsize{30}{30}\sffamily\bfseries#1}};
		\end{tikzpicture}};
\end{tikzpicture}}
\addcontentsline{toc}{part}{\texorpdfstring{%
	\setlength\fboxsep{0pt}%
	\noindent\protect\colorbox{theme!40}{\strut\protect\parbox[c][.7cm]{\linewidth}{\Large\sffamily\protect\centering #1\quad\mbox{}}}}{#1}}%
\@endpart}
\def\@endpart{\vfil\newpage
\if@twoside
\if@openright
\null
\thispagestyle{empty}%
\newpage
\fi
\fi
\if@tempswa
\twocolumn
\fi}


%%% THEOREM STYLES
\renewcommand{\qedsymbol}{$\square$}% QED square style
\usepackage[most]{tcolorbox}
\newcounter{allResults}\numberwithin{allResults}{section}
% THEOREM-LIKE
% Theorem; red border all round, grey background
\newtcbtheorem[number within=section]{tcbtheo}{Theorem}%
{fonttitle=\sffamily\bfseries\color{resultColor},
      separator sign dash, description color=black,
      top=7pt, bottom=7pt,
      sharp corners, frame hidden, boxrule=0pt, boxsep=0pt, breakable,
      lines before break = 4,
      enhanced, borderline={\the\tcbBorderWidth}{0pt}{resultColor},
      parbox=false,
      colback=resultBackColor,
      coltext=black,
      attach title to upper={\\},
      subtitle style = {enhanced jigsaw, boxrule=0pt, bottomrule=0.5pt, bottom=2pt,
      attach title to upper={\\},
      frame code = {\path[black,draw] ([xshift=\the\tcbBorderWidth]frame.south west) -- ([xshift=-\the\tcbBorderWidth]frame.south east);}, 
      opacityback=0, fontupper=\color{black}}}{thm}
\NewDocumentEnvironment{theorem}{ O{} O{} }
    {\begin{tcbtheo}{#1}{#2}}{\end{tcbtheo}}
% Corollary; red border left, no background
\newtcbtheorem[use counter from=tcbtheo]{tcbcorollary}{Corollary}%
{fonttitle=\sffamily\bfseries\color{resultColor},
      separator sign dash, description color=black,
      top=2pt, bottom=2pt,
      sharp corners,
      frame hidden, boxrule=0pt, boxsep=0pt, breakable,
      lines before break = 3,
      enhanced, borderline west={\the\tcbBorderWidth}{0pt}{resultColor},
      parbox=false,
      colback=black!0,
      coltext=black,
      attach title to upper={\\},
      subtitle style = {enhanced jigsaw, boxrule=0pt, bottomrule=0.5pt, bottom=2pt,
      attach title to upper={\\},
      frame code = {\path[black,draw] ([xshift=\the\tcbBorderWidth]frame.south west) -- (frame.south east);}, 
      opacityback=0, fontupper=\color{black}}}{cor}
\NewDocumentEnvironment{corollary}{ O{} O{} }
    {\begin{tcbcorollary}{#1}{#2}}{\end{tcbcorollary}}
% Lemma; red border left, grey background
 \newtcbtheorem[use counter from=tcbtheo]{tcblemma}{Lemma}%
     {fonttitle=\sffamily\bfseries\color{resultColor},
      separator sign dash, description color=black,
      top=2pt, bottom=2pt,
      sharp corners,
      frame hidden, boxrule=0pt, boxsep=0pt, breakable,
      enhanced, borderline west={\the\tcbBorderWidth}{0pt}{resultColor},
      parbox=false,
      colback=resultBackColor,
      coltext=black,
      attach title to upper={\\},
      subtitle style = {enhanced jigsaw, boxrule=0pt, bottomrule=0.5pt, bottom=2pt,
      attach title to upper={\\},
      frame code = {\path[black,draw] ([xshift=\the\tcbBorderWidth]frame.south west) -- (frame.south east);}, 
      opacityback=0, fontupper=\color{black}}}{lem}
\NewDocumentEnvironment{lemma}{ O{} O{} }
    {\begin{tcblemma}{#1}{#2}}{\end{tcblemma}}
% Proposition; red border left, no background
\newtcbtheorem[use counter from=tcbtheo]{tcbproposition}{Proposition}%
     {fonttitle=\sffamily\bfseries\color{resultColor},
      separator sign dash, description color=black,
      top=2pt, bottom=2pt,
      sharp corners,
      frame hidden, boxrule=0pt, boxsep=0pt, breakable,
      enhanced, borderline west={\the\tcbBorderWidth}{0pt}{resultColor},
      parbox=false,
      coltext=black,
      colback=black!0,
      attach title to upper={\\},
      subtitle style = {enhanced jigsaw, boxrule=0pt, bottomrule=0.5pt, bottom=2pt,
      attach title to upper={\\},
      frame code = {\path[black,draw] ([xshift=\the\tcbBorderWidth]frame.south west) -- (frame.south east);}, 
      opacityback=0, fontupper=\color{black}}}{prop}
\NewDocumentEnvironment{proposition}{ O{} O{} }
    {\begin{tcbproposition}{#1}{#2}}{\end{tcbproposition}}
% DEFINITION-LIKE
% Definition; blue border left, no background
\newtcbtheorem[use counter from=tcbtheo]{tcbdef}{Definition}%
     {fonttitle=\sffamily\bfseries\color{definitionColor},
      separator sign dash, description color=black,
      top=2pt, bottom=2pt,
      sharp corners,
      frame hidden, boxrule=0pt, boxsep=0pt, unbreakable,
      enhanced, borderline west={\the\tcbBorderWidth}{0pt}{definitionColor},
      parbox=false,
      colback=black!0,
      coltext=black,
      attach title to upper={\\},
      subtitle style = {enhanced jigsaw, boxrule=0pt, bottomrule=0.5pt, bottom=2pt,
      attach title to upper={\\},
      frame code = {\path[black,draw] ([xshift=\the\tcbBorderWidth]frame.south west) -- (frame.south east);}, 
      opacityback=0, fontupper=\color{black}}}{def}
\NewDocumentEnvironment{definition}{ O{} O{} }
    {\begin{tcbdef}{#1}{#2}}{\end{tcbdef}}
% Notation; blue border left, no background
\newtcbtheorem[use counter from=tcbtheo]{tcbnotation}{Notation}%
     {fonttitle=\sffamily\bfseries\color{definitionColor},
      separator sign dash, description color=black,
      top=2pt, bottom=2pt,
      sharp corners,
      frame hidden, boxrule=0pt, boxsep=0pt, unbreakable,
      enhanced, borderline west={\the\tcbBorderWidth}{0pt}{definitionColor},
      parbox=false,
      colback=black!0,
      coltext=black,
      attach title to upper={\\},
      subtitle style = {enhanced jigsaw, boxrule=0pt, bottomrule=0.5pt, bottom=2pt,
      attach title to upper={\\},
      frame code = {\path[black,draw] ([xshift=\the\tcbBorderWidth]frame.south west) -- (frame.south east);},
      opacityback=0, fontupper=\color{black}}}{not}
\NewDocumentEnvironment{notation}{ O{} O{} }
    {\begin{tcbnotation*}{#1}{#2}}{\end{tcbnotation*}}
% EXAMPLE-LIKE
% Example; green border left, no background
\newtcbtheorem[use counter from=tcbtheo]{tcbexample}{Example}%
     {fonttitle=\sffamily\bfseries\color{exampleColor},
      separator sign dash, description color=black,
      top=2pt, bottom=2pt,
      sharp corners,
      frame hidden, boxrule=0pt, boxsep=0pt, breakable,
      enhanced, borderline west={\the\tcbBorderWidth}{0pt}{exampleColor},
      parbox=false,
      colback=exampleBackColor,
      coltext=black,
      attach title to upper={\\},
      subtitle style = {enhanced jigsaw, boxrule=0pt, bottomrule=0.5pt, bottom=2pt,
      attach title to upper={\\},
      frame code = {\path[black,draw] ([xshift=\the\tcbBorderWidth]frame.south west) -- (frame.south east);}, 
      opacityback=0, fontupper=\color{black}}}{exam}
\NewDocumentEnvironment{example}{ O{} O{} }
    {\begin{tcbexample}{#1}{#2}}{\end{tcbexample}}
% Examples
\newtcbtheorem[use counter from=tcbtheo]{tcbexamples}{Examples}%
     {fonttitle=\sffamily\bfseries\color{exampleColor},
      separator sign dash, description color=black,
      top=2pt, bottom=2pt,
      sharp corners,
      frame hidden, boxrule=0pt, boxsep=0pt, breakable,
      enhanced, borderline west={\the\tcbBorderWidth}{0pt}{exampleColor},
      parbox=false,
      colback=exampleBackColor,
      coltext=black,
      parbox=false,
      attach title to upper={\\},
      subtitle style = {enhanced jigsaw, boxrule=0pt, bottomrule=0.5pt, bottom=2pt,
      attach title to upper={\\},
      frame code = {\path[black,draw] ([xshift=\the\tcbBorderWidth]frame.south west) -- (frame.south east);}, 
      opacityback=0, fontupper=\color{black}}}{exam}
\NewDocumentEnvironment{examples}{ O{} O{} }
    {\begin{tcbexamples}{#1}{#2}}{\end{tcbexamples}}
\NewDocumentEnvironment{examtemize}{ O{} O{} }
    {\begin{tcbexamples}{#1}{#2}\vspace*{-15pt}\begin{itemize}}{\end{itemize}\end{tcbexamples}}
% PROBLEM
\newtcbtheorem[use counter from=tcbtheo]{tcbproblem}{Problem}%
     {fonttitle=\sffamily\bfseries\color{black},
      separator sign dash, description color=black,
      top=2pt, bottom=2pt,
      sharp corners,
      frame hidden, boxrule=0pt, boxsep=0pt, breakable,
      enhanced, borderline west={\the\tcbBorderWidth}{0pt}{problemColor},
      parbox=false,
      colback=problemBackColor,
      coltext=black,
      parbox=false,
      attach title to upper={\\},
      subtitle style = {enhanced jigsaw, boxrule=0pt, bottomrule=0.5pt, bottom=2pt,
      attach title to upper={\\},
      frame code = {\path[black,draw] ([xshift=\the\tcbBorderWidth]frame.south west) -- (frame.south east);}, 
      opacityback=0, fontupper=\color{black}}}{exam}
\NewDocumentEnvironment{problem}{ O{} O{} }
    {\begin{tcbproblem}{#1}{#2}}{\end{tcbproblem}}
% EXERCISE
\newtcbtheorem[use counter from=tcbtheo]{tcbexercise}{Exercise}%
    {fonttitle=\sffamily\bfseries\color{black},
        separator sign dash, description color=black,
        top=2pt, bottom=2pt,
        sharp corners,
        frame hidden, boxrule=0pt, boxsep=0pt, breakable,
        enhanced, borderline west={\the\tcbBorderWidth}{0pt}{exerciseColor},
        parbox=false,
        colback=exerciseBackColor,
        coltext=black,
        parbox=false,
        attach title to upper={\\},
        subtitle style = {enhanced jigsaw, boxrule=0pt, bottomrule=0.5pt, bottom=2pt,
        attach title to upper = {},
        frame code = {\path[black,draw] ([xshift=\the\tcbBorderWidth]frame.south west) -- (frame.south east);},
opacityback=0, fontupper=\color{black}}}{exer}
\NewDocumentEnvironment{exercise}{ O{} O{} }
    {\begin{tcbexercise}{#1}{#2}}{\end{tcbexercise}}

    \newtcbtheorem[use counter from=tcbtheo]{tcbalgo}{Algorithm}%
    {fonttitle=\sffamily\bfseries\color{black},
        separator sign dash, description color=black,
        top=2pt, bottom=2pt,
        sharp corners,
        frame hidden, boxrule=0pt, boxsep=0pt, breakable,
        enhanced, borderline west={\the\tcbBorderWidth}{0pt}{exerciseColor},
        parbox=false,
        colback=exerciseBackColor,
        coltext=black,
        parbox=false,
        attach title to upper={\\},
        subtitle style = {enhanced jigsaw, boxrule=0pt, bottomrule=0.5pt, bottom=2pt,
        attach title to upper = {\\},
        frame code = {\path[black,draw] ([xshift=\the\tcbBorderWidth]frame.south west) -- (frame.south east);},
opacityback=0, fontupper=\color{black}}}{algo}
\NewDocumentEnvironment{algorithm}{ O{} O{} }
{\begin{tcbalgo}{#1}{#2}}{\end{tcbalgo}}
\NewDocumentEnvironment{algorithm*}{ O{} O{} }
{\begin{tcbalgo*}{#1}{#2}}{\end{tcbalgo*}}




% HYPERLINKS    
\usepackage[hidelinks]{hyperref}
%\hypersetup{hidelinks,backref=true,pagebackref=true,hyperindex=true,colorlinks=false,breaklinks=true,urlcolor= theme,bookmarks=true,bookmarksopen=false,pdftitle={Title},pdfauthor={Author}}
\usepackage{bookmark}
\bookmarksetup{open,
               numbered,
               addtohook={\ifnum\bookmarkget{level}=0 % chapter
                          \bookmarksetup{bold}%
                          \fi
                          \ifnum\bookmarkget{level}=-1 % part
                          \bookmarksetup{color=theme,bold}%
                          \fi
                         }
                     }

% ENVIRONMENT TWEAKS                     
% Fixes proof environment to start on new line
\makeatletter
\renewenvironment{proof}[1][\proofname]{\par
\pushQED{\qed}%
\normalfont \topsep6\p@\@plus6\p@\relax
\trivlist
\item[\hskip\labelsep
\itshape
#1\@addpunct{.}]
\leavevmode}{%
\popQED\endtrivlist\@endpefalse
}
\makeatother

\newcommand{\slide}[1]{\subsubsection{#1}}


\lstnewenvironment{python}
{\vspace{5pt}\begin{center}\rule{0.9\textwidth}{1pt}\end{center}\lstset{language=Python,linewidth=0.9\linewidth,xleftmargin=0.1\linewidth}}
{\vspace{-10pt}\begin{center}\rule{0.9\textwidth}{1pt}\end{center}}
\newcommand{\pythoncode}[1]{\lstinline[language=Python]{#1}}
\newcommand{\code}[1]{\texttt{#1}}
